\documentclass[12pt,russian]{article}
\usepackage{graphicx}
\usepackage{verbatim}
\usepackage[utf8]{inputenc}
\usepackage[russian]{babel}
\usepackage[T2A]{fontenc}
\usepackage{hyperref}

\hypersetup{
pdfauthor = {Ivan Bunin},
pdftitle = {Zhizn Arsenyeva},
pdfsubject = {ebook},
pdfkeywords = {Bunin, Arseniev, fb2pdf}}

\setcounter{secnumdepth}{0} 

\begin{document}

\title{Жизнь Арсеньева}
\author{Иван Бунин}
\maketitle

\begin{quotation}
"Вещи и дела,  аще  не  написанiи  бываютъ,  тмою  покрываются и  гробу безпамятства предаются, написавшiи же яко одушевленiи ..."
\end{quotation}

\section{Книга Первая}

\subsection{I}

Я родился полвека тому назад, в средней России,  в деревне, в отцовской усадьбе. 

У нас нет чувства своего начала и конца. И очень жаль, что мне сказали, когда именно я родился. Если бы не сказали, я  бы теперь и понятия не имел о  своем  возрасте, \cdash--- тем более, что я еще совсем не ощущаю его бремени, \cdash--- и, значит,  был  бы избавлен от  мысли,  что мне будто бы полагается лет  через десять или двадцать умереть. А родись я и живи на необитаемом  острове, я бы даже и о самом  существовании смерти не подозревал. "Вот было бы счастье  !" \cdash--- хочется прибавить мне. Но кто  знает? Может быть, великое несчастье. Да и правда  ли,  что  не подозревал бы? Не рождаемся ли  мы с чувством смерти? А если нет, если бы не подозревал, любил ли бы я жизнь так, как люблю и любил?

О роде  Арсеньевых, о его происхождении мне почти ничего  не  известно. Что  мы вообще знаем! Я  знаю только то, что в Гербовнике род наш  отнесен к тем,  "происхождение  коих теряется  во  мраке  времен".  Знаю, что  род наш "знатный, хотя и захудалый" и что я всю {8} жизнь чувствовал эту  знатность, гордясь и радуясь, что я не из тех, у кого  нет ни рода, ни племени. В Духов день призывает Церковь за литургией "сотворить память всем от века умершим". Она возносит в этот день прекрасную и полную глубокого смысла молитву:

\begin{quotation}
     \cdash--- Вси рабы Твоя,  Боже,  упокой во дворех Твоих и в недрех Авраама, \cdash--- от  Адама даже до  днесь послужившая Тебе чисто отцы и  братiи наши, други и сродники!
\end{quotation}

Разве случайно сказано здесь о служении? И разве не радость чувствовать свою связь,  соучастие "с отцы  и  братiи наши, други и  сродники",  некогда совершавшими  это  служение? Исповедовали наши древнейшие пращуры учение  "о чистом,  непрерывном  пути  Отца  всякой  жизни",  переходящего от  смертных родителей к смертным чадам  их \cdash---  жизнью бессмертной, "непрерывной", веру в то,  что это  волей Агни  заповедано  блюсти  чистоту, непрерывность  крови, породы, дабы не был "осквернен", то есть прерван этот "путь", и что с каждым рождением  должна  все более  очищаться кровь рождающихся  и  возрастать  их родство, близость с ним, единым Отцом всего сущего.

Среди моих  предков  было, верно,  не  мало  и  дурных. Но  все  же  из поколения в поколение наказывали мои предки друг другу помнить и блюсти свою кровь: будь достоин во  всем своего благородства. И как передать те чувства, с которыми я  смотрю порой  на  наш родовой  герб? Рыцарские доспехи, латы и шлем с страусовыми перьями. Под ними щит. И на лазурном поле его, в середине \cdash--- перстень, эмблема верности и вечности, к которому сходятся сверху и снизу своими остриями три рапиры с крестами-рукоятками.

В стране, заменившей мне родину, много есть городов, подобных тому, что дал  мне  приют,  некогда  {9}  славных,   а  теперь  заглохших,  бедных,  в повседневности живущих  мелкой жизнью. Все же над этой жизнью всегда \cdash--- и не даром \cdash--- царит какая-нибудь серая башня времен  крестоносцев, громада собора с бесценным порталом, века  охраняемым  стражей  святых изваяний, и петух на кресте, в небесах, высокий Господний глашатай, зовущий к небесному Граду.


\end{document}
